\chapter{Moving Data}



\section{Defining Data Elements}


\subsection{Stack, Heap, .bss, .data, .text}
\begin{itemize}
    \item Stack is used to store local variables, such as parameters in a function or variables created in{}. It is allocated by system in C.
    \item Heap is allocated mostly by memory allocate functions. This memory is managed by programmer.
    \item .data is used to store initialized static data. It has a exact memory rom. These data are stored both in object files and execuation files.
    \item .bss is used to store uninitialized static data. In object files, it is only a symbol to show needing memory space. In memory image, it is allocated with required memory space.
    \item .text is used to store unchangable data like source code.
\end{itemize}
\subsection{.data}
\begin{itemize}
    \item Any data elements declared in this section are reserved in memory and can be read or written to by instructions in the assembly language program.
    \item After the directive is declared, a default value (or values) must be defined.
    \item The lowest memory value contains the first data element.
\end{itemize}

\subsubsection{Code Example}
\begin{lstlisting}
.section .data                   #.rodata(read only)
msg:
    .ascii "This is a message"   #.asciz(string used 
                                 #for C functions 
                                 #like printf)
value
    .int 32, 43, 432             #define many data in
                                 #one line is allowed
.equ PI 3.1415926                #constant value
\end{lstlisting}



\mbox{}\\


\colorbox[gray]{0.8}
{
	\parbox[t]{35em}
	{
		.section is a Pseudo-operation. It can be used to tell assembler how to organize data. It is not a instruction. There are some types like .text, .data predefined.
	}
}


\subsection{.bss}
\begin{tabular}[c]{l|l}
    \hline
        Directives & Discription\\
    \hline
        .comm & Declares a common memory area for data that is not initialized\\
        .lcomm & Declares a local common memory area for data that is not initialized\\
    \hline
\end{tabular}
\begin{lstlisting}
.section .bss
.comm buffer 1000
\end{lstlisting}



\section{Moving Data Elements}
\begin{itemize}
	\item Moving immediate data to registers and memory
	\begin{lstlisting}
movl $0, %eax	#Register
movl $0, height	#Memory
	\end{lstlisting}
	\item Moving data between registers
	\begin{lstlisting}
movl %eax, %ebx
movw %ax, %cx
	\end{lstlisting}
	\item Moving data between memory and registers
	\begin{itemize}
		\item Moving data values from memory to a register
		\begin{lstlisting}
value:
	.int 1
movl value, %eax	#$eax=1
movl $value, %eax	#$eax=(address of value)
		\end{lstlisting}
		\item Moving data values from a register to memory
		\begin{lstlisting}
movl %eax, value
		\end{lstlisting}
		\item Using indexed memory locations
		
		base\_address(offset\_address, index, size)
		\begin{lstlisting}
.section .data
output:
.asciz "The value is %d\n"
values:
.int 10, 15, 20, 25, 30, 35, 40, 45, 50, 55, 60
.section .text
.globl _start
_start:
nop
movl $0, %edi
movl $4, %ebx
loop:
	movl values(%ebx, %edi, 4), %eax
	pushl %eax
	pushl $output
	call printf
	addl $8, %esp
	inc %edi
	cmpl $11, %edi
	jne loop
	movl $0, %ebx
	movl $1, %eax
	int $0x80
		\end{lstlisting}
		\item Using indirect addressing with registers
		
		You can get the memory
		location address of the data value by placing a dollar sign (\$) in front of the label in the instruction.
		\begin{lstlisting}
movl $value, %eax
movl %ebx, -4(%eax)
		\end{lstlisting}
	\end{itemize}
\end{itemize}
\colorbox[gray]{0.8}
{
	\parbox{36em}
	{
		\emph{How to compile 32bits code in 64bits}
		\begin{itemize}
			\item as --32 -gstabs -o a.o source.s
			\item ld -m elf\_i386 --oformat elf32-i386 -dynamic-linker /usr/lib32/ld-linux.so.2 -o a a.o /usr/lib32/libc.so
		\end{itemize}
	}
}

\section{Conditional Move Instructions}
\begin{tabular}{c|c|c}
	\hline
	Data Type & eflags & Instructions\\
	\hline
	signed & SF OF ZF & ge/nl l/nge le/ng o no s ns\\
	unsigned & CF ZF PF & a/be ae/nb nc b/ae c be/na e/z ne/nz p/pe np/po\\
\end{tabular}
