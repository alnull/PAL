\chapter{What Is Assembly Language}
\section{Why to use assembly language}
\begin{itemize}
    \item To simplify the instruction codes(historical reason)
    \item To improve the effiency of usig high level language(present reason)
\end{itemize}
\section{Instruction code design format}
\begin{itemize}
    \item \emph{Software = Data + Code}
        
        So design two pointers: data pointer and instruction pointer
        \begin{itemize} 
            \item Instruction pointer infers what code is next in line to be processed
            \item Data pointer infers the start address of stack
        \end{itemize}
    \item \emph{Instruction code format}
        \begin{itemize}
            \item Optional instruction prefix: modify the opcode behavior
            \item Operation code: define function or task
            \item Optional modifier: extra information to locate address
            \item Optional data element
        \end{itemize}
\end{itemize}
\section{The relationship between al and hll}
Many languages can be compiled or interpreted to al source. Al source files then can be assembled to machine code.
\section{Assembly Language}
\begin{itemize}
    \item use opcode mnemonic to facilitate writing
    \item two ways to define data
        \begin{itemize}
            \item use memory locations
            \item use the stack
        \end{itemize}
    \item directives: similiar to keywords in C/C++
        \begin{itemize}
            \item .bss: buffer
            \item .text: code
            \item .data: data
        \end{itemize}
\end{itemize}
